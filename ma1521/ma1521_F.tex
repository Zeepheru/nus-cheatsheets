\documentclass[10pt]{article}
\usepackage{multicol}
\usepackage{calc}
\usepackage{ifthen}
\usepackage{amsmath,amsthm,amsfonts,amssymb}
\usepackage{color,graphicx,overpic}
\usepackage{hyperref}
\usepackage{listings}
\usepackage{color}
\usepackage{physics}


\usepackage[a4paper, left=1cm, right=1cm, top=0.75cm, bottom=0.75cm, portrait]{geometry}

% Turn off header and footer
\pagestyle{empty}

% Redefine section commands to use less space
\makeatletter
\renewcommand{\section}{\@startsection{section}{1}{0mm}%
                                {-1ex plus -.5ex minus -.2ex}%
                                {0.5ex plus .2ex}%x
                                {\normalfont\large\bfseries}}
\renewcommand{\subsection}{\@startsection{subsection}{2}{0mm}%
                                {-1explus -.5ex minus -.2ex}%
                                {0.5ex plus .2ex}%
                                {\normalfont\normalsize\bfseries}}
\renewcommand{\subsubsection}{\@startsection{subsubsection}{3}{0mm}%
                                {-1ex plus -.5ex minus -.2ex}%
                                {1ex plus .2ex}%
                                {\normalfont\small\bfseries}}
\makeatother

% Define BibTeX command
\def\BibTeX{{\rm B\kern-.05em{\sc i\kern-.025em b}\kern-.08em
    T\kern-.1667em\lower.7ex\hbox{E}\kern-.125emX}}

% print only section numbers
\setcounter{secnumdepth}{1}


\setlength{\parindent}{0pt}
\setlength{\parskip}{0pt plus 0.5ex}

%My Environments
\newtheorem{example}[section]{Example}


% -----------------------------------------------------------------------

\begin{document}
\raggedright
\footnotesize


% ADDITIONAL STUFF
% dividing line :)
\newcommand{\divider}{\noindent\makebox[\linewidth]{\rule{\columnwidth}{0.4pt}}}

\begin{center}
{\large \textbf{MA1521 Finals Cheatsheet AY23/24}\\{by tysng (AY19/20) \texttt{github.com/tysng/ma1521-cheatsheet}, modified by Zeepheru for AY23/24}}
\end{center}
\hrulefill

\begin{multicols}{2}

% multicol parameters
% These lengths are set only within the two main columns
%\setlength{\columnseprule}{0.25pt}
\setlength{\premulticols}{1pt}
\setlength{\postmulticols}{1pt}
\setlength{\multicolsep}{1pt}
\setlength{\columnsep}{2pt}

% ------------------------------ACTUAL CONTENT-----------------------------------

\section{MF26++}

\subsection{Trigo}
\begin{gather*}
    \sin^2 x + \cos^2 x = 1 \\
   \sin (A \pm B) = \sin A \cos B \pm \cos A \sin B \\
   \cos (A \pm B) = \cos A \cos B \mp \sin A \sin B \\
   \tan (A \pm B) = \frac{\tan A \pm \tan B}{1 \mp \tan A \tan B} \\
   \sin 2A = 2 \sin A \cos A \\
   \cos 2A = \cos ^ 2 A - \sin ^ 2 A = 2\cos^2A - 1 = 1 - 2\sin^2A \\
   \tan 2A = \frac{2 \tan A}{1- \tan^2 A} \\
   \sin P + \sin Q = 2 \sin \frac{1}{2} (P + Q) \cos \frac{1}{2} (P - Q) \\
   \sin P - \sin Q = 2 \cos \frac{1}{2} (P + Q) \sin \frac{1}{2} (P - Q) \\
   \cos P + \cos Q = 2 \cos \frac{1}{2} (P + Q) \cos \frac{1}{2} (P - Q) \\
   \cos P - \cos Q = -2 \sin \frac{1}{2} (P + Q) \sin \frac{1}{2} (P - Q)
\end{gather*}

\subsection{Derivatives}
\begin{multicols*}{2}
    \begin{align*} 
        & e^x \to e^x \; | \; \ln x \to \frac{1}{x} \\
        & \csc x \to - \csc x \cot x \\
        & \sec x \to \sec x \tan x \\
        & \tan x \to \sec^2 x \\
        & \cot x \to -\csc^2 x
    \end{align*}
    \begin{align*}
        & \asin x \to \frac{1}{\sqrt{1-x^2}} \\
        & \acos x \to -\frac{1}{\sqrt{1-x^2}} \\
        & \atan x \to \frac{1}{1+x^2}
    \end{align*}
\end{multicols*}



\subsection{Integrals}
Take note of the absolute sign, and always remember to $+c$.\\
$x$ can be replaced by $x + b, \; \forall \: b \in \mathbb{R}$.
\begin{multicols*}{2}
    \begin{align*}
        & \frac{1}{x} \to \ln \abs{x} \\
        % & \sec^2 \to \tan \: ; \: \csc^2 \to \cot \\
        & \tan x \to \ln |\sec x| \\
        & \cot x \to \ln |\csc x| \\
        & \sec x \to \ln |\sec x + \tan x| \\
        & \csc x \to \ln |\csc x + \cot x|
    \end{align*}
    \begin{align*}
        & \frac{1}{x^2 + a^2}  \to \frac{1}{a} \atan (\frac{x}{a}) \\ 
        & \frac{1}{\sqrt{a^2 - x^2}}  \to \asin (\frac{x}{a}) \\
        & \frac{-1}{\sqrt{a^2 - x^2}}  \to \acos (\frac{x}{a}) \\
        & \frac{1}{a^2 - x^2} \to \frac{1}{2a} ln \left\vert\frac{x+a}{x-a}\right\vert
    \end{align*}
\end{multicols*}

\textbf{Improper integrals}: infinite bounds of integration (I) or functions that become infinite within the interval (II). \\
Replace the bounds with limits where necessary.
\[ \textbf{eg, I: } \int^{\infty}_{-\infty} f(x) dx = \lim_{a \to -\infty}\int^{b}_{a} f(x) dx + \lim_{c \to \infty}\int^{c}_{b} f(x) dx \]
\[ \textbf{eg, II: } \int^{b}_{a} f(x) dx = \lim_{c \to a^+}\int^{b}_{c} f(x) dx \]

\textbf{Applications} \\
Disk method (about $x$-axis): $\pi \int_{a}^{b} f(x)^2 - g(x)^2 dx $ \\
Shell method (about $y$-axis): $2\pi \int_{a}^{b}x | f(x) - g(x)| dx $ \\
Arc length: $\int_{a}^{b} \sqrt{1 + f'(x)^2} dx $
\smallskip

\textbf{Partial Fractions}
\[ \frac{px+q}{(ax+b)^2} = \frac{X}{ax+b} + \frac{Y}{(ax+b)^2} \implies \begin{cases}
    X = \frac{p}{a} \\
    Y = q - \frac{pb}{a} \\
\end{cases}
\]

\[ \frac{px+q}{(ax+b)(cx+d)} = \frac{X}{ax+b} + \frac{Y}{cx+d} \implies \text{solve } \begin{cases}
    cX + aY = p \\
    dX + bY = q \\
\end{cases}
\]

\divider
\section{Basics}
\subsection{Critical Points}
\begin{itemize}
   \item Interior point where $f'(a) = 0$
   \item Interior points where $f'(a)$ doesn't exist
\end{itemize}
Endpoints of a domain can be \textbf{absolute} min/max, but \textbf{not} local.

\subsection{Misc Theorems}
$f$ is cont. on $[a, b]$ and diff. on $(a, b)$. \\
\textbf{Rolle's Theorem: }$f(a) = f(b) \implies \exists c \in (a, b) \text{ s.t.} f'(c) = 0$ \\

\textbf{Mean Value Theorem: }$\exists c \in (a, b) \text{ s.t.} f'(c) = \frac{f(b) - f(a)}{b-a}$

\subsection{L'Hopital's Rule}
The $\frac{0}{0}$ form: (1) $f$ and $g$ are differentiable in a neighborhood of $x_0$, 
(2) $f(x_0) = g(x_0) = 0$, (3) $g'(x) \neq 0$ except possibly at $x_0$
\begin{gather*}
   \lim_{x\to x_0} \frac{f(x)}{g(x)} = \lim_{x\to x_0} \frac{f'(x)}{g'(x)}
\end{gather*}
E.g. $\lim_{x\to 0} \frac{3x - \sin x }{x} = \frac{3-\cos x }{1} \rvert_{x = 0} = 2$

The $\frac{\infty}{\infty}$ form: when $x \to a$, $f(x), g(x) \to \infty$,
and both differentiable,
\begin{gather*}
   \lim_{x\to a} \frac{f(x)}{g(x)} = \lim_{x\to a} \frac{f'(x)}{g'(x)}
\end{gather*}

Else, change to these two forms. \\
(e.g $\lim_{x\to 0^+} x \cot x = \lim_{x\to 0^+} \frac{x}{\tan x} 
= \lim_{x\to 0^+} \frac{1}{\sec^2 x} = 1$ )

\[ \text{other forms: }\infty - \infty, 1 ^ \infty, \infty ^ 0, 0^0 \]

\subsection{Fundamental Theorem of Calculus}
\[ \left. \int_a^b f(x) dx = F(b) - F(a) \: \right\vert \: \frac{d}{dx} \int_{c}^{x} f(t) dt = f(x) \]
   


\divider
\section{Series}
\textbf{Geometric Series:} $S_n = a \cdot \frac{1-r^n}{1-r}$, $r\neq 1$, converges $\iff |r| < 1$.

\begin{align*}
    & \textbf{P-Series:} \sum_{n=1}^{\infty} \frac{1}{n^p} \begin{cases}
        p>1:\text{converges} \\
        0 \leq p \leq 1:\text{diverges}
    \end{cases} ( \textbf{harmonic series: } p = 1)
\end{align*}

\[
\textbf{n-th term test: } \begin{cases}
    \sum^{\infty}_{n=1}a_n \text{ converges} \implies \lim_{n \to \infty} a_n = 0 \\
    \lim_{n \to \infty} a_n \neq 0 \implies \sum^{\infty}_{n=1}a_n \text{ diverges}
\end{cases}
\]

\textbf{Integral Test:} $a_n = f(n), f$ is continuous, positive, and decreasing.
\begin{align*}
    \text{Then} \sum_{n=b}^{\infty} a_n \; \text{converges} \iff \int_{b}^{\infty} f(x) dx \; \text{converges}
\end{align*}

\textbf{Comparison Test:} Compare with an appropriate $p$-series, harmonic series, or geometric series.
\begin{align*}
    & L = \begin{matrix}
        \textbf{Ratio Test:} \lim_{n\to\infty}\abs{\frac{a_{n+1}}{a_n}} \\
        \textbf{Root Test:} \lim_{n\to\infty}\sqrt[n]{|a_n|} \\
    \end{matrix} \begin{cases}
        0 \leq L < 1 \implies \text{absolutely convergent} \\
        L > 1 \implies \text{divergent} \\
        L = 1 \implies \text{inconclusive}
    \end{cases}
\end{align*}
\[\textbf{Absolute Convergence: } \text{both} \sum^{\infty}_{n=1}|a_n| \; \text{and} \;  \sum^{\infty}_{n=1}a_n \; \text{converge.}\]
\begin{align*}
    & \textbf{Alt. Series Test: } b_n >0, b_n \; \text{is decreasing}, \lim_{n\to\infty}b_n = 0 \\
    & \implies \sum_{n=1}^{\infty}(-1)^{n-1}b_n \text{ converges.}
\end{align*}


\subsection{Power Series}
\[\sum_{n = 0}^{\infty} c_n(x-a)^n = c_0 + c_1(x-a) + c_2(x-a)^2 + \cdots,\]
where $a$ is the center of the power series \\ 

% Convergence: $n\to\infty, S_n \to k$, possibilities:
% \begin{enumerate}
%    \item $\sum c_n(x-a)^n$ converges at $x=a$ and diverges elsewhere
%    \item $h \in \mathbb{Z}$ that the series only converges in $(a-h, a + h)$
%    \item converges for every $x$
% \end{enumerate}

To find a \textbf{power series representation} for some $f$, transform to:
\[\frac{1}{1-\Box} = \sum_{n = 0 }^{\infty} \Box^n, \abs{\Box} < 1 \; \left\vert \; \frac{1}{1+x} = \sum_{n = 0 }^{\infty}(-x)^n \right. \]

\[\text{e.g. } \frac{1}{x^2+3x+2} = \frac{1}{x+1}-\frac{1}{2}\frac{1}{1+x/2} = 
\sum_{n = 0 }^{\infty}(-x)^n - \frac{1}{2} \sum_{n = 0 }^{\infty}(-\frac{x}{2})^n\]

\textbf{Finding Radius of Convergence} \\
\begin{itemize}
    \item If $c_n \neq 0 \; \forall c_n$, use ratio/root test, $R = \frac{1}{L}$.
    \item Then check for convergence at the endpoints $a\pm R$ to determine open/closed interval.
\end{itemize}

\[\text{if not, find }\lim_{n\to\infty} \abs{\frac{u_{n+1}}{u_n}} < 1\]

and \textbf{transform} it to the form of $\abs{x-a} < b$; $a$ is the center, $R = b$.

\[\text{Note that} \begin{cases}
    L = 0 \; (\text{converges } \forall x) \implies R = \infty \\
    L = \infty \; (\text{converges at one } x) \implies R = 0
\end{cases}\]

\begin{align*}
    & \text{If } R > 0, \forall x, |x-a| < R \\
    & f(x) = \sum_{n=0}^{\infty} c_n(x-a)^n \text{ is differentiable.} \\
    & f'(x) = \sum_{n=1}^{\infty} nc_n(x-a)^{n-1}, \; \int f(x)dx = \frac{c_n(x-a)^{n+1}}{n+1} + C \; \;
\end{align*}



\subsection{Taylor Series}
\[f(x) = \sum_{k=0}^{\infty} \frac{f^{(k)}(a)}{k!} (x-a)^k\]
\begin{multicols*}{2}
    \begin{align*}
       & e^x = \sum_{n=0}^{\infty} \frac{x^n}{n!} \\
       & \sin x = \sum_{n=0}^{\infty} \frac{(-1)^n x ^{2n+1}}{(2n+1)!} \\
       & \cos x = \sum_{n=0}^{\infty} \frac{(-1)^n x ^{2n}}{(2n)!}
    \end{align*}
        
    \begin{align*}
        & \ln (1+x )= \sum_{n=1}^{\infty} \frac{(-1)^{n-1}x^n}{n} \\
        & \tan^{-1} x = \sum_{n=0}^{\infty} \frac{(-1)^n x^{2n+1}}{2n+1}
    \end{align*}
\end{multicols*}


% \subsection{Finding a specific high order derivative}
% \begin{enumerate}
%    \item given $\int f dx$
%    \item evaluate f in polynomial form and integrate the polynomial form
%    \item Compare the coefficient with the item that contains $f^{(100)}(0)$ in the Taylor expansion
% \end{enumerate}







\divider
\section{Vectors}
\begin{multicols*}{2}
    Use the form: $\vec{a} = \langle x, y, z \rangle$ instead of $\textbf{a}$.\\ 
    Angle: $\cos \theta = \frac{\textbf{a} \cdot \textbf{b}}{\norm{\textbf{a}} \norm{\textbf{b}}}$ \\
    Proj of $\textbf{b}$ on $\textbf{a}$: $\text{proj}_\textbf{a}\textbf{b} =  \frac{\textbf{a} \cdot \textbf{b}}{\textbf{a} \cdot \textbf{a}} \textbf{a}$.\\
    Length: $\text{comp}_\textbf{a}\textbf{b} =  \frac{\textbf{a} \cdot \textbf{b}}{\norm{\textbf{a}} }$ 
    \begin{center}
        \underline{Cross Product}
    \end{center}
    \[\textbf{a} \times \textbf{b} = \begin{vmatrix}
    \textbf{i} && \textbf{j} && \textbf{k} \\
    a_1 && a_2 && a_3 \\
    b_1 && b_2 && b_3 
    \end{vmatrix}\]
    \\ \; \\ \; 
\end{multicols*}
Find foot of perpendicular on \textbf{plane}: create line using normal vector and point, then find intersection. \\
Find foot of perpendicular of point $C$ on \textbf{line} $l$:
\begin{itemize}
    \item[] let $\vec{OF}$ be on $l$. $\implies \vec{CF} = \vec{OF} - \vec{OC} \implies \vec{CF} \cdot l = 0$. Find $\lambda$.
\end{itemize}


\divider
\section{Multivariable Functions}
\subsection{1-Var, Vector functions}
\[\textbf{r}(t) = \langle f(t), g(t), h(t) \rangle\]
\textbf{Arc Length: }$S = \int_a^b \sqrt{f'(t)^2 + g'(t)^2 + h'(t)^2} dt = \int_a^b \norm{\vec{r}'t} dt$ \\

\subsection{2-Var}

\textbf{Cylinder: }$\exists$ a plane $P$ s.t. all parallel planes intersect the surface.
\begin{itemize}
    \item[] All eqns with \textbf{1 var missing} are cylinders. 
\end{itemize}
\begin{multicols*}{2}
    \begin{center}
        \textbf{Elliptic paraboloid} about $z$:
    \end{center}
    \[\frac{x^2}{a^2} + \frac{y^2}{b^2} = \frac{z}{c}\]
    \;
    \begin{center}
        \textbf{Ellipsoid}:
    \end{center}
    \[\frac{x^2}{a^2} + \frac{y^2}{b^2} + \frac{z^2}{c^2} = 1\]
    \;
\end{multicols*}
\textbf{Level curve: } $f(x, y) = k$ for some $k$. \\
\textbf{Contour plot: } $f(x, y) = k$ for equally spaced $k \in {k_1, k_2, \cdot}$.

\subsection{Partial Differentiation}
\textbf{Clairaut's Theorem: } $f_{xy} = f_{yx}$ \\ 
\textbf{Tangent Plane:} $z = f(a, b) + f_x(a, b)(x - a) + f_y(a, b)(y-b)$

\[\textbf{Chain Rule: } \frac{dz}{dt} = \frac{\partial f}{\partial x}\frac{dx}{dt} + \frac{\partial f}{\partial y}\frac{dy}{dt}\]

Same idea if $x, y$ are expressed in more variables, $x = x(t, t_1, t_2, \ldots)$ \\

\subsection{Directional Derivative} 
(similar for functions in 3 variables)
\begin{align*}
    & \textbf{Gradient Vector: }\nabla f = \langle f_x, f_y \rangle \\
    & D_{\textbf{u}}f(a,b) = f_x(a, b) \cdot {u}_1 + f_y(a,b) \cdot {u}_2 = \nabla f(a,b) \cdot \textbf{u}, \\
    & \text{unit vector }{\textbf{u}} = \langle u_1, u_2 \rangle \\
    & \text{Thus, } D_{\textbf{u}}f(a,b) = \norm{\nabla f(a,b)} \cos \theta
\end{align*}


$f$ increases most rapidly in $\nabla f(a,b)$, decreases most rapidly in $ -\nabla f(a,b)$

\textbf{Max value} of $D_u f(a,b) = \norm{\nabla f(a,b)}$, when $\vec{u}$ and $\nabla f$ in the same direction, since $\cos \theta = 0$

Increment in $f$ (approx.): $\Delta f \approx [D_{\vec{u}} f(\vec{p})] (\Delta t)$, where $p$ is the origin, $u$ is the unit direction.
\\

% \textbf{Finding D$_{u}$f} \\
% \begin{enumerate}
%    \item Find the direction $\textbf{p}$
%    \item Find the unit vector $\textbf{u} = \frac{\textbf{p}}{\norm{\textbf{p}}} $
%    \item Find $\nabla f$, then find $D_u f = \nabla f \cdot \textbf{u}$
% \end{enumerate}



\subsection{2-Var Critical Points}
A point of $f$ that satisfies either is a critical point:
\begin{enumerate}
   \item $f_x (a,b) = 0$ and $f_y(a,b) = 0$
   \item $f_x (a,b)$ or $f_y(a,b)$ doesn't exist
\end{enumerate}

Perform Second Derivative Test: let $f_x(a,b) = 0$ and $f_y(a,b) = 0$
$$ D = f_{xx}(a,b) f_{yy} (a,b) - f_{xy} (a,b)^2 $$
\[\begin{cases}
    D > 0, f_{xx} >0, f\text{ has a local minimum at }(a,b) \\
    D > 0, f_{xx} <0, f\text{ has a local maximum at }(a,b) \\
    D < 0, f\text{ has a saddle point at }(a,b) \\
    D = 0,\text{ no conclusion}
\end{cases}\]



\divider
\section{Double Integrals}
For $R = \{ (x, y) : a \leq x \leq b, c \leq y \leq d \},$
\[
\int\!\int_R\! f(x, y)\, \mathrm{d}A = \int_c^d\!\int_a^b\! f(x, y)\, \mathrm{d}x \; \mathrm{d}y = \int_a^b\!\int_c^d\! f(x, y)\, \mathrm{d}y \; \mathrm{d}x
\]

if $f(x, y) = g(x)h(y)$, then
\[\int\!\int_R f(x,y) \, \mathrm{d}A = (\int_a^b\!g(x)\, \mathrm{d}x) (\int_c^d\!h(y)\, \mathrm{d}y)\]

\subsection{General Regions}
Express horizontal/vertical bounds as a function $g(x)$ or $h(y)$ \\
\[ 
\int_a^b\!\int_{g_1(x)}^{g_2(x)}\!f(x, y)\, \mathrm{d}y \; \mathrm{d}x \qquad
\int_c^d\!\int_{h_1(y)}^{h_2(y)}\!f(x, y)\, \mathrm{d}x \; \mathrm{d}y
\]

\subsection{Polar Coordinates}
$R = \{(r, \theta) : a \leq r \leq b$, $\alpha \leq \theta \leq \beta\}$
$$
\int\!\int_R\! f(x, y)\, \mathrm{d}A = 
\int_\alpha^\beta\!\int_a^b\!f(r\cos\theta, r\sin\theta)r\, \mathrm{d}r\mathrm{d}\theta
$$
Note: $x^2 + y^2 = r^2$.
\subsection{Surface Area}
$$S = \int\!\int_R\! \sqrt{(\frac{\partial z}{\partial x})^2 + (\frac{\partial z}{\partial y})^2 + 1 }\, \mathrm{d}A$$


\divider
\section{Ordinary Differential Equations}
\subsection{Separable Equations}
\[
\frac{dy}{dx} = f(x)g(y)\implies \int \frac{1}{g(y)} dy = \int f(x) dx + C
\]

\subsection{Reduction to Separable Form (by Substitution)}
Let $v = y / x \implies y = xv \rightarrow y' = v + xv'$, transform equations of $y' = g(\frac{y}{x})$ 
to $v + xv' = g(v) $ such that
\begin{gather*}
   \frac{dv}{g(v) - v} = \frac{dx}{x}
\end{gather*}
Similarly, $y' = f(ax + by + c)$ can be solved by $u = ax + by + c$

\subsection{Linear First Order ODE}
To solve $y' + P(x)y = Q(x)$: 
\begin{align*}
    & \text{Find integration factor } I(x) = e^{\int P(x) dx} \\
    & \text{Then, } y = \frac{1}{I(x)} \int Q(x) I(x) dx
\end{align*}


\subsection{Bernoulli Equation}
\begin{gather*}
    y' + P(x)y = Q(x) y^n, \; n \neq 0, 1 \\
    \text{Let } u = y^{1-n}. \\
    \implies u' + (1-n)p(x)u = (1-n)q(x) \text{ (Linear 1st order)}
\end{gather*}

When $n > 0, y(x) = 0$ is a solution.  

% \subsection{Homogeneous Linear Second Order DE}
% For $y'' + a y' + by = 0$, the characteristic equation is $\lambda^2 + a \lambda + b = 0$

% Find $\Delta = a ^2 -4b$:

% \begin{enumerate}
%    \item $\Delta > 0$, $y = c_1e^{\lambda_1x} + c_2e^{\lambda_2x}$
%    \item $\Delta = 0$, $y = (c_1 + c_2 x) e^{-\frac{ax}{2}} $
%    \item $\Delta < 0$, it has two complex roots;$ \lambda_1 = \alpha + \beta i, \lambda_2 = \alpha -\beta i$; $y = c_1 e^{\alpha x} \cos \beta x + c_2 e^{\alpha x} \sin \beta x$
% \end{enumerate}
% where,
% \begin{gather*}
%    \lambda_1 = \frac{1}{2} (-a+\sqrt{a^2-4b}) \\
%    \lambda_2 = \frac{1}{2} (-a-\sqrt{a^2-4b})
% \end{gather*}

% NOT TESTED FOR 23/24S1

% \section{Modeling}
% \subsection{Population Growth}
% Malthus's Model: not an accurate representation
% \begin{gather*}
%    \frac{dN}{dt} = kN, k = B - D \\
%    N(t) = N_0 e^{kt} 
% \end{gather*}

% \subsection{Logistic Model}
% Assume $D = sN$, where $s$ is a constant:
% $$ \frac{dN}{dt} = BN - DN = BN - sN^2 $$

% The curve approaches carrying capacity $N = B/S$; point of inflection is at $N = B/2s$

% $$ N = \frac{N_{\infty}}{1 + (\frac{N_{\infty}}{N_0} - 1) e^{-Bt}}, N_{\infty} = \frac{B}{s} $$

% \subsection{Harvesting}
% Basic harvesting model: $\frac{dN}{dt} = BN - sN^2 - E$, where E is fish catched/ year.

% Desirable result: $E < \frac{B^2}{4s}$, approaches the second root $\beta_2 = \frac{B + \sqrt{B^2 - 4Es}}{2s}$, when $dN/dt = 0$ 

% \subsection{Strategies}
% \begin{itemize}
%    \item When given $dx/dt$, find $x$ that $dx/dt = 0$, draw out the axis, determine the sign of $dx/dt$ within
%       each region, and find the flow (+ to the right, - to the left)
%    \item To find E, draw the graph without E and find the line of symmetry; use the product of the roots to find E;


% \end{itemize}


% \section{PDE}
% For a PDE in the form of,
% \begin{gather*}
%    u_x = f(x) g(y) u_y
% \end{gather*}
% Substitute $u(x,y) = X(x) Y(y)$ in the PDE, usually 
% $$u_x = X'Y, u_y = XY', u_{xy} = X'Y'$$
% and arrange the DE into a from in which $X', Y'$ both has power of 1;

% Let both sides be $k$, or let them be $k, 1/k$; and solve $X, Y$ in $k, c$;



% You can even have references
% Nah
% \rule{0.3\linewidth}{0.25pt}
% \scriptsize
% \bibliographystyle{abstract}
% \bibliography{refFile}
\end{multicols}
\end{document}