\documentclass[10pt]{article}
\usepackage{multicol}
\usepackage{calc}
\usepackage{ifthen}
\usepackage{amsmath,amsthm,amsfonts,amssymb}
\usepackage{color,graphicx,overpic}
\usepackage{hyperref}
\usepackage{listings}
\usepackage{color}

% Use LuaLaTeX
\usepackage{emoji}


\usepackage[a4paper, left=1cm, right=1cm, top=0.75cm, bottom=0.75cm, portrait]{geometry}

% Turn off header and footer
\pagestyle{empty}

% Redefine section commands to use less space
\makeatletter
\renewcommand{\section}{\@startsection{section}{1}{0mm}%
                                {-1ex plus -.5ex minus -.2ex}%
                                {0.5ex plus .2ex}%x
                                {\normalfont\large\bfseries}}
\renewcommand{\subsection}{\@startsection{subsection}{2}{0mm}%
                                {-1explus -.5ex minus -.2ex}%
                                {0.5ex plus .2ex}%
                                {\normalfont\normalsize\bfseries}}
\renewcommand{\subsubsection}{\@startsection{subsubsection}{3}{0mm}%
                                {-1ex plus -.5ex minus -.2ex}%
                                {1ex plus .2ex}%
                                {\normalfont\small\bfseries}}
\makeatother

% Define BibTeX command
\def\BibTeX{{\rm B\kern-.05em{\sc i\kern-.025em b}\kern-.08em
    T\kern-.1667em\lower.7ex\hbox{E}\kern-.125emX}}

% print only section numbers
\setcounter{secnumdepth}{1}


\setlength{\parindent}{0pt}
\setlength{\parskip}{0pt plus 0.5ex}

%My Environments
\newtheorem{example}[section]{Example}


% -----------------------------------------------------------------------

\begin{document}
\raggedright
\footnotesize


% ADDITIONAL STUFF
% dividing line :)
% \newcommand{\divider}{\noindent\makebox[\linewidth]{\rule{\columnwidth}{0.4pt}}}

\begin{center}
{\large \textbf{CS1101S Finals Reference AY23/24}\\{by Zeepheru (AY23/24)}}
\end{center}
\hrulefill

\begin{multicols}{2}

\section*{Orders of Growth}

\textit{(Limits used for succinctness, not correct.)} \\ 
\textbf{Big Theta},  \textbf{Big Omega}, and \textbf{Big O}:
\begin{align*}
    & \theta(g(n))\; \iff \exists \, k_1, k_2 \in \mathbb{Z}^+ \; \exists \, n_0 \in \mathbb{R}  \\
    & \hspace{1cm} ( \forall \, n > n_0 (k_{1} \cdot g(n) \leq r(n) \leq k_{2} \cdot g(n))) \\
    & O(g(n))\; \iff \exists \, k \in \mathbb{Z}^+  ( \lim_{n \to \infty } (k \cdot g(n) \geq r(n) )) \\
    & \Omega(g(n))\; \iff \exists \, k \in \mathbb{Z}^+  ( \lim_{n \to \infty } (k \cdot g(n) \leq r(n) ))
\end{align*}

\textbf{Order (smol to big):} $1$, $\log n$, $n$, $n \log n$, $n^{2}$, $n^{3}$, $2^{n}$, $3^{n}$, $n^{n}$ \\
\smallskip

Note: $r(n)$ has OOGs $\theta(r(n))$, $O(r(n))$, and $\Omega(r(n))$. \\

{\textbf{Common Recurrence Relations}}
\begin{align*}
    T(n) & = O(1) + T(n-1) \implies O(n) \\
        & = O(\log n) + T(n-1) \implies O(n \log n) \\
        & = O(n) + T(n-1) \implies O(n^2) \\
        & = O(1) + 2T(n-1) \implies O(2^n) \\
        & = O(1) + T(\frac{n}{2}) \implies O(\log n) \\
        & = O(n) + 2T(\frac{n}{2}) \implies O(n \log n) \\ 
        & = O(n) + T(\frac{n}{2}) \implies O(n) \\
        & = O(1) + 2T(\frac{n}{2}) \implies O(n)
\end{align*}

Generally, $T(n) = O(n^k) + T(n-1)  \implies O(n^{k+1})$ \\

\hrulefill
\section*{Lists}

\textbf{\emoji{star} Check if x is in list xs}
\begin{verbatim}
    if (!is_null(member(x, xs))) { }
\end{verbatim}

\textbf{Remove Duplicates}
\begin{verbatim}
    function remove_duplicates(xs) {
        return accumulate(
                    (curr, wish) => 
                        pair(curr, 
                             filter(x => x !== curr, wish)),
                    null, xs);
    }
\end{verbatim}

\textbf{Permutations}
\begin{verbatim}
    function permutations(ys) {
        // list => list of lists
        return is_null(ys)
            ? list(null)
            : accumulate(append, null,
                map(x => map(p => pair(x, p),
                             permutations(remove(x, ys))),
                    ys));
    }
\end{verbatim}

\hrulefill

\section*{Arrays}

\textbf{\emoji{star} Reverse Index}
\begin{verbatim}
    A[len - i - 1];
\end{verbatim}

\textbf{List-array conversion}
\begin{verbatim}
    function list_to_array(xs) {
        const A = [];
        let i = 0;
        while (!is_null(xs)) {
            A[i] = head(xs);
            xs = tail(xs);
            i = i + 1;
        }
        
        return A;
    }
    
    function array_to_list(A) {
        function helper(i) {
            if (A[i] === undefined) {
                return null;
            } else {
                return pair(A[i], helper(i + 1));
            }
        }
        
        return helper(0);
    }
\end{verbatim}

\hrulefill
\section*{Reminders :)}
Return a \verb|block|:
\begin{verbatim}
return {...; return x; };  
\end{verbatim}

\verb|stream_tail| returns \verb|null| if nothing left, \verb|stream_ref| returns \verb|undefined|.\\
\smallskip

On \verb|equal()|:
\begin{verbatim}
    const a = pair(null, null);
    const b1 = pair(a, a);
    const b2 = pair(a, a);
    
    equal(b1, b2); # returns TRUE
\end{verbatim}

\subsubsection{QRFs}
\begin{itemize}
    \item \verb|integers_from(start)| - Stream.
    \item \verb|for_each(f, xs)| - Maps functions in place.
    \item \verb|build_stream/list(f, n)| - $0$ to $n-1$.
    \item \verb|enum_stream/list(start, end)| - Includes \verb|start| and \verb|end|.
    \item \verb|eval_stream(s, n)| - Generates list of first $n$ stream values. 
\end{itemize}


% multicol parameters
% These lengths are set only within the two main columns
%\setlength{\columnseprule}{0.25pt}
\setlength{\premulticols}{1pt}
\setlength{\postmulticols}{1pt}
\setlength{\multicolsep}{1pt}
\setlength{\columnsep}{2pt}

% ------------------------------ACTUAL CONTENT-----------------------------------

\end{multicols}
\end{document}