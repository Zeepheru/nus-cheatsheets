\documentclass[a4paper]{article} \usepackage[backend=biber, style=numeric, sorting=none]{biblatex}
\usepackage[a4paper, left=1cm, right=1cm, top=0.75cm, bottom=0.75cm, landscape]{geometry}
% \usepackage{showframe}
\usepackage{multicol}
\usepackage{blindtext}
\usepackage{amsmath}
\usepackage{amsfonts}
\usepackage{amssymb}

\usepackage{listings}

\begin{document}
\setlength\parindent{0pt} %TODO this somehow messes up paragraph vertical spacing as well? 
\scriptsize
% \tiny
\pagenumbering{gobble}

% dividing line :)
\newcommand{\divider}{\noindent\makebox[\linewidth]{\rule{\columnwidth}{0.4pt}}}

\begin{center}
{\large CS1101S Midterms Cheatsheet AY23/24}\\{by chrisgzf (AY19/20), modified by Zeepheru for AY23/24}
\end{center}
    \begin{multicols*}{4}

{\small\textbf{Recursive/Iterative}}: Check if there are deferred operations

\begin{verbatim}
function fact_iter(n) {
    function mult_remaining(counter , product) {
        return counter === 1
            ? product
            : mult_remaining(counter - 1, product
            * counter);
    }
    return mult_remaining(n, 1);
}

function fib(n) {
    function f(n, k, x, y) {
        return (k > n) ? y : f(n, k + 1, y, x + y);
    }
    return (n < 2) ? n : f(n, 2, 0, 1);
}

function gcd(a, b) {
    return b === 0 ? a : gcd(b, a % b);
}

function cc(amount , kinds_of_coins) {
    return amount === 0
        ? 1
        : amount < 0 || kinds_of_coins === 0
            ? 0
            : cc(amount - first_denomination(kinds_
                 of_coins), kinds_of_coins) +
              cc(amount , kinds_of_coins - 1);
}
\end{verbatim}
{\small\textbf{Lists}}: A list is either null or a pair whose tail is a list.
\\ \\
A list of a certain type is either null or a pair whose head and tail are of that type.

\begin{verbatim}

function reverse(xs) {
    function rev(original, reversed) {
        return is_null(original)
            ? reversed
            : rev(tail(original),
                  pair(head(original), reversed));
    }
    return rev(xs ,null);
}

function append_iter(xs, ys){
    // iterative process
    function app(xs, ys, c) {
        return is_null(xs)
        ? c(ys)
        : app(tail(xs), ys, 
              x => c(pair(head(xs), x))
              );
    }
    return app(xs, ys, x => x);
}




function remove_duplicates(lst) {
    return is_null(lst)
        ? null
        : pair(head(lst), remove_duplicates(
            filter(x => !equal(x, head(lst)),
                   tail(lst))));
}
\end{verbatim}
% Passing the deferred operation as a func. in an extra arg. is called “Continuation-Passing Style” (CPS). \\

% TREES %
{\small\textbf{Trees}}: A tree of certain data items is a list whose elements are such data items, or trees of such data items.

\begin{verbatim}
function map_tree(f, tree) {
    return map(sub_tree =>
               !is_list(sub_tree)
                   ? f(sub_tree)
                   : map_tree(f, sub_tree)
               , tree);
}

function flatten_tree(xs) {
    function h(xs, prev) {
        return is_null(xs)
            ? prev // end of list or tree
            : is_list(xs)
                ? append(flatten(xs), prev) //list
                : pair(xs, prev); // leaf
    }
    return accumulate(h, null, xs);
}
\end{verbatim}

Besides the base case, these operations consider two cases. One, when the element is itself a tree, and another when it is not. 
\\ \\

% BINARY TREES %
{\small\textbf{Binary Trees}}: A binary tree of a certain type is null or a list with three elements, whose first element is of that type and whose second and third elements are binary trees of that type.
\\ \\
% BINARY SEARCH TREES %
{\small\textbf{Binary Search Trees}}: A binary search tree of Strings is a binary tree of Strings where all entries in the left subtree are smaller than its value and all entries in the right subtree are larger than its value.

\begin{verbatim}
function insert(bst, item) {
    if (is_empty_tree(bst)) {
        return make_tree(item, make_empty_tree(),
        				 make_empty_tree());
    } else {
        if (item < entry(bst)) {
            // smaller than i.e. left branch
            return make_tree(entry(bst),
                       insert(left_branch(bst),
                              item),
                       right_branch(bst));
        } else if (item > entry(bst)) {
            // bigger than entry i.e. right branch
            return make_tree(entry(bst),
                       left_branch(bst),
                       insert(right_branch(bst),
                              item));


                              
        } else {
            // equal to entry.
            // BSTs should not contain duplicates
            return bst;
        }
    }
}

function find(bst, name) {
    return is_empty_tree(bst)
        ? false
        : name === entry(bst)
            ? true
            : name < entry(bst)
                ? find(left_branch(bst), name)
                : find(right_branch(bst), name);
}
\end{verbatim}

% PERMUTATIONS %
{\small\textbf{Permutations \& Combinations}}
\begin{verbatim}
function permutations(s) {
    return is_null(s)
        ? list(null)
        : accumulate(append, null,
                     map(x => map(p => pair(x, p),
                     permutations(remove(x, s))),
                     s));
}

function subsets(s) {
    return accumulate(
        (x, s1) => append(s1,
                   map(ss => pair(x, ss), s1)),
        list(null),
        s);
}
function choose(n, r) {
    if (n < 0 || r < 0) {
        return 0;
    } else if (r === 0) {
        return 1;
    } else {
        // Consider the 1st item, there are 2 choices:
        // To use, or not to use
        // Get remaining items with wishful thinking
        const to_use = choose(n - 1, r - 1);
        const not_to_use = choose(n - 1, r);
        
        return to_use + not_to_use;
    }
}
function combinations(xs, r) {
    if ( (r !== 0 && xs === null) || r < 0) {
        return null;
    } else if (r === 0) {
        return list(null);
    } else {
        const no_choose = combinations(tail(xs), r);
        const yes_choose = combinations(tail(xs),
                                        r - 1);
        const yes_item = map(x => pair(head(xs), x),
                             yes_choose);
        return append(no_choose, yes_item);
    }
}
function makeup_amount(x, coins) {
    if (x === 0) {
        return list(null);
    } else if (x < 0 || is_null(coins)) {
        return null;
    } else {
        // Combinations that do not use the head coin.
        const combi_A = makeup_amount(x, tail(coins));
        // Combinations that do not use the head coin
        // for the remaining amount.
        const combi_B = makeup_amount(x - head(coins),
                                      tail(coins));
        // Combinations that use the head coin.
        const combi_C = map(x => pair(head(coins), x),
                            combi_B);
        return append(combi_A, combi_C);
    }
}
\end{verbatim}

\divider
\\ \\
{\small\textbf{Orders of Growth}}
\\
\textit{(Limits used for succinctness, not correct.)}\\ \\
\textbf{Big Theta},  \textbf{Big Omega}, and \textbf{Big O}:
\\ \\
$\theta(g(n))\; \iff \exists \, k_1, k_2 \in \mathbb{Z}^+ \; \exists \, n_0 \in \mathbb{R}  \\
( \forall \, n > n_0 (k_{1} \cdot g(n) \leq r(n) \leq k_{2} \cdot g(n))) $
\\ \\
$O(g(n))\; \iff \exists \, k \in \mathbb{Z}^+  ( \lim_{n \to \infty } (k \cdot g(n) \geq r(n) )) $
\\ \\
$\Omega(g(n))\; \iff \exists \, k \in \mathbb{Z}^+  ( \lim_{n \to \infty } (k \cdot g(n) \leq r(n) )) $
\\ \\
\textbf{Order (smol to big):} \\ $1 < \log n < n < n \log n < n^{2} < n^{3} < 2^{n} < 3^{n} < n^{n}$ \\

Note: $r(n)$ has OOGs $\theta(r(n))$, $O(r(n))$, and $\Omega(r(n))$. \\

{\textbf{Common Recurrence Relations}}
\begin{align*}
    T(n) & = O(1) + T(n-1) \implies O(n) \\
        & = O(\log n) + T(n-1) \implies O(n \log n) \\
        & = O(n) + T(n-1) \implies O(n^2) \\
        & = O(1) + 2T(n-1) \implies O(2^n) \\
        & = O(1) + T\left(\frac{n}{2}\right) \implies O(\log n) \\
        & = O(n) + 2T\left(\frac{n}{2}\right) \implies O(n \log n) \\ 
        & = O(n) + T\left(\frac{n}{2}\right) \implies O(n) \\
        & = O(1) + 2T\left(\frac{n}{2}\right) \implies O(n)
\end{align*}

Generally, $T(n) = O(n^k) + T(n-1)  \implies O(n^{k+1})$ \\

\vfill\null
\columnbreak


{\small\textbf{Insertion sort}} takes elements from left to right, and \textit{inserts} them into correct positions in the sorted portion of the list (or array) on the left. This is analagous to how most people would arrange playing cards.
\\ \\
\textbf{Time Complexity:} $\Omega(n)\ O(n^{2})$
\begin{verbatim}
function insert(x, xs) {
    return is_null(xs)
        ? list(x)
        : x <= head(xs)
            ? pair(x, xs)
            : pair(head(xs), insert(x, tail(xs)));
}

function insertion_sort(xs) {
    return is_null(xs)
        ? xs
        : insert(head(xs),
                 insertion_sort(tail(xs)));
}
\end{verbatim}

{\small\textbf{Selection sort}} picks the smallest element from a list (or array) and puts them in order in a new list.
\\ \\
\textbf{Time Complexity:} $\Omega(n^{2})\ O(n^{2})$
\begin{verbatim}
function selection_sort(xs) {
    if (is_null(xs)) {
        return xs;
    } else {
        const x = smallest(xs);
        return pair(x,
            selection_sort(remove(x, xs)));
    }
}

function smallest(xs) {
    function h(xs, min) {
        return xs === null
            ? min
            : head(xs) < min
                ? h(tail(xs), head(xs))
                : h(tail(xs), min);
    }
    return h(xs, head(xs));
}
\end{verbatim}
\vfill\null
\columnbreak
{\small\textbf{Quicksort}} is a divide-and-conquer algorithm. \texttt{Partition} takes a pivot, and positions all elements smaller than the pivot on one side, and those larger on the other. The two `sides' are then partitioned again.
\\ \\
\textbf{Time Complexity:} $\Omega(n log n)\ O(n^{2})$
\begin{verbatim}
function partition(xs, p) {
    function h(xs, lte, gt) {
        if (is_null(xs)) {
            return pair(lte, gt);
        } else {
            const first = head(xs);
            return first <= p
                ? h(tail(xs), pair(first, lte), gt)
                : h(tail(xs), lte, pair(first, gt));
        }
    }
    return h(xs, null, null);
}


function quicksort(xs) {
    if (is_null(xs) || is_null(tail(xs))) {
        return xs;
    } else {
        const pivot = head(xs);
        const splits = partition(tail(xs), pivot);
        const smaller = quicksort(head(splits));
        const bigger = quicksort(tail(splits));
        return append(smaller, pair(pivot, bigger));
    }
}
\end{verbatim}
{\small\textbf{Mergesort}} is a divide-and-conquer algorithm.
\\ \\
\textbf{Time Complexity:} $\Omega(n log n)\ O(n log n)$
\begin{verbatim}
function take(xs, n) {
    return n === 0
        ? null
        : pair(head(xs),
               take(tail(xs), n - 1));
}
function drop(xs, n) {
    return n === 0
        ? xs
        : drop(tail(xs), n - 1);
}

function merge(xs, ys) {
    if (is_null(xs)) { 
        return ys;
    } else if (is_null(ys)) { 
        return xs;
    } else {
        const x = head(xs);
        const y = head(ys);
        return (x < y) 
            ? pair(x, merge(tail(xs), ys))
            : pair(y, merge(xs, tail(ys)));
    }
}






function merge_sort(xs) {
    if (is_null(xs) || is_null(tail(xs))) {
        return xs;
    } else {
        const mid = math_floor(length(xs) / 2);  
        return merge(merge_sort(take(xs, mid)), 
                     merge_sort(drop(xs, mid)));
    }
}
\end{verbatim}
\divider
\\
\\
{\small\textbf{Personal Additions}}

\begin{verbatim}
function accumulate(op, initial, xs) {
    return is_null(xs) 
            ? initial
            : op(head(xs), 
                accumulate(op, initial, tail(xs)));
}

function remove_duplicates(lst) {
    return accumulate(
                    (x, y) => is_null(member(x, y)) 
                                ? pair(x, y) : y,
                    null, lst);
}

function subsets(xs) {
    if (is_null(xs)) {
        return list(null);
    } else {
        const no_subsets = subsets(tail(xs));
        const yes_subsets = map(
                    x => pair(head(xs), x),
                              subsets(tail(xs)));
        
        return append(no_subsets, yes_subsets);
    }
}

function subsets_acc(xs) {
    return accumulate(
                (x, y) =>
                    append(y, map(t => pair(x, t), y)),
                list(null), xs);
}

\end{verbatim}

\vfill\null
\columnbreak

{\small\textbf{Misc Notes}}
\\ \\
They do like to ask on \textbf{active lists}. In all prior implementations, \textbf{active lists} are \textbf{FUNCTIONS}. So all operations on them should make use of this $\implies$ use the indexes. 
\begin{itemize}
    \item Also note that an active list called on an index returns a \textbf{list} of length 1.
\end{itemize}

\\ \\
Also more generally, remember to use the \textbf{properties} of the data structures given! Eg: list is sorted, left tree is smaller than right tree, etc. 
\\ \\
Master Theorem for \textbf{recurrence relations} (Citation Needed): \\
For $T(n) = aT(\frac{n}{b}) + \Theta({n^d}), \; a \geq 1, \; b >  1, \; d \geq 0$,
\begin{align*}
& a < b^d \implies \Theta(n^d) \\
& a = b^d \implies \Theta(n^d \log n) \\
& a > b^d \implies \Theta(n^{\log_b a})
\end{align*}

{\textbf{DRAW THE STARTING ARROW / NAME REF FOR BOX \& POINTER DIAGS!}} 
\\ \\ 
Something I tend to miss is boolean results and combining them with bool operators for more compact solutions. 

\end{multicols*}
\end{document}